\documentclass[12pt]{extarticle}
\usepackage[utf8]{inputenc}
\usepackage{mathtools} % matrices, cases
\usepackage{amsfonts} % natural/whole/real/etc. number sets
\usepackage{geometry}
\geometry{verbose,a4paper,tmargin=0.8cm,bmargin=0.8cm,lmargin=0.8cm,rmargin=0.8cm}
\setlength{\parindent}{0cm}

\begin{document}
	\huge \underline{Proprietatea de universalitate} \\
	{\large
		% \underline{Teoremă} (proprietatea de universalitate a inelelor de polinoame) \\
		% R inel comutativ \\
		% $f : R \to S$ morfism de inele comutative si $\alpha \in S$ \\
		% Atunci există un morfism de inele $\overline{f} : R[x] \to S$
	}
\end{document}